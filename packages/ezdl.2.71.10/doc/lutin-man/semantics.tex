
\newcommand{\trans}[1]{\mbox{$\stackrel{#1}{\rightarrow}$}}
\newcommand{\vanish}{\raisebox{-0mm}{$\;\rotatebox{90}{$\scriptstyle \hookleftarrow$}$}}
\newcommand{\diewith}[1]{\raisebox{-0mm}{$\;\rotatebox{90}{$\scriptstyle \hookrightarrow$}^{#1}$}}
\newcommand{\catch}[2]{\mbox{$[#2]_{#1}$}}
\newcommand{\catchdo}[3]{\mbox{$[#1\stackrel{#2}{\hookrightarrow}#3]$}}
\newcommand{\try}[1]{\mbox{$[#1]_{\delta}$}}
\newcommand{\trydo}[2]{\catchdo{#1}{\delta}{#2}}
\newcommand{\deadlock}{\diewith{\delta}}
\newcommand{\merge}[2]{\mbox{$#1\;\&\;#2$}}
\newcommand{\WLOOP}[2]{\mbox{$#1^{(\omega_c, \omega_s)}_{#2}$}}

\newcommand{\RUN}{\mbox{\it Run}}
\newcommand{\RUNE}{\mbox{${\cal R}_e$}}

\newcommand{\ACTIONS}{\mbox{$\cal A$}}
\newcommand{\TRACES}{\mbox{$\cal T$}}
\newcommand{\CONSTRAINTS}{\mbox{$\cal C$}}
\newcommand{\TERMINATIONS}{\mbox{$\cal X$}}

\newcommand{\ITE}[3]{(#1)?\;#2\,:\,#3}
\newcommand{\LET}{\mbox{\it let}}
\newcommand{\IN}{\mbox{\it in}}
\newcommand{\WHERE}{\mbox{\it where}}

%%%%%%%%%%%%%%%%%%%%%%%%%%%%%%%%%%%%%%%%%%%%%%%%%%%%%%%%%%%%%%%%%%%%%%%%%
%%%%%%%%%%%%%%%%%%%%%%%%%%%%%%%%%%%%%%%%%%%%%%%%%%%%%%%%%%%%%%%%%%%%%%%%%
\section{Semantics}
\label{semantics}

%%%%%%%%%%%%%%%%%%%%%%%%%%%%%%%%%%%%%%%%%%%%%%%%%%%%%%%%%%%%%%%%%%%%%%%%%
\subsection{Abstract syntax}

The semantics is defined according to the following abstract syntax,
where:
\begin{minitemize}
\item we only consider binary priority choice and parallel composition,
since they are left-associative,
\item we define the empty-behaviour ($\varepsilon$)
and the empty-behaviour filter ($t\setminus\varepsilon$),
which are not available in the concrete syntax, but useful for defining
the semantics,
\item random loops are {\em normalized} by expliciting their
weight functions:
\begin{minitemize}
\item the stop function $\omega_s$ takes the number of iteration already performed
and returns the relative weight of the ``stop'' choice, 
\item the continue function $\omega_c$ takes the number of iteration already performed
and returns the relative weight of the ``continue'' choice. 
\end{minitemize}
These functions  are completly determined  by the ``profile''  of the
loop in the  concrete syntax (interval or average,  together with the
corresponding  static  arguments).  See~\S\ref{loop-profiles}  for  a
precise definition of these weight functions.
\item the actual number of (already) performed iterations is syntacticaly attached
to the loop; this is convenient to define the semantics in terms
of rewriting. In the main statement, this flag is obviously set
to $0$.
\end{minitemize}

\begin{minipage}{70mm}
\begin{tabular}{rl}
empty behaviour: & $\varepsilon$ \\
atomic constraint: & $c$ \\
raise: & $\diewith{x}$ \\
sequence: & $t \;\cdot\; t'$ \\
priority: & $t \;\succ\; t'$ \\
parallel: & $\merge{t}{t'}$
\end{tabular}
\end{minipage}\begin{minipage}{6cm}
\begin{tabular}{rl}
empty filter: & $t\setminus\varepsilon$ \\
catch: & $\catchdo{t}{x}{t'}$ \\
%choice: & $t/w \;|\; t'/w'$ \\
choice: & $|_{i=1}^n\;\;t_i/w_i$\\
random loop: & $\WLOOP{t}{i}$ \\
priority loop: & $t^*$
\end{tabular}
\end{minipage}

\TRACES\ denotes the set of trace expressions, and \CONSTRAINTS\ the
set of constraints.
%% \begin{definition} **)
%% \end{definition} **)

%%%%%%%%%%%%%%%%%%%%%%%%%%%%%%%%%%%%%%%%%%%%%%%%%%%%%%%%%%%%%%%%%%%%%%%%%
\subsection{The run function}
\label{run-function}
The semantics of an execution step is given by a function
taking an environment $e$ and a (trace) expression $t$:
$\RUN(e,t)$.
%The semantics accepts any programs, comprising 
%incorrect ones that can generates instantaneous loops.
%When an instantaneous loop is detected a
%fatal run-time errors is raised.

This function returns an {\em action} which is either:
\begin{minitemize}
\item a transition $\trans{c}{n}$, which means that $t$ produces
a constraint $c$ and rewrite itself in the (next) trace $n$,
\item a termination $\diewith{x}$, where $x$ is a termination flag
which is either $\varepsilon$ (normal termination),
$\delta$ (deadlock) or some user-defined exception.
\end{minitemize}

\ACTIONS\ denotes the set of actions, and \TERMINATIONS\ denotes the
 set of termination flags.
%% \begin{definition} **)
%% \end{definition} **)

The run function is inductively defined using a 
recursive function $\RUNE(t, g, s)$ where the parameters
$g$ and $s$ are continuation functions returning actions.
\begin{minitemize}
\item $g: \CONSTRAINTS \times \TRACES \mapsto \ACTIONS$
is the {\em goto} function, defining how
a local transition should be treated according to the calling context.
\item $s: \TERMINATIONS \mapsto \ACTIONS$
is the {\em stop} function, defining how
a local termination should be treated according to the calling context.
\end{minitemize}

At the top level, \RUNE\ is simply called with the trivial continuations:
\begin{eqnarray}
\RUN(e,t) & = & \RUNE(t,\;\;
\lambda(c,v).\trans{c}{v},\;\;
\lambda x. \diewith{x}
)
\end{eqnarray}


%%%%%%%%%%%%%%%%%%%%%%%%%%%%%%%%%%%%%%%%%%%%%%%%%%%%%%%%%%%%%%%%%%%%%%%%%
\subsection{The recursive run function}

\subsubsection{Basic traces.}

The empty behavior raises the termination flag in the current context:
\[
\RUNE(\varepsilon, g, s) = s(\varepsilon)
\]
A raise statement terminates with the corresponding flag:
\[
\RUNE(\diewith{x}, g, s) = s(x)
\]
A constraint generates a goto or raises a deadlock, depending on its 
satisfiability in the environment:
\[
\RUNE(c, g, s) = \ITE{e\models c}{g(c, \varepsilon)}{s(\delta)}
\]

\subsubsection{Sequence.}

\[
\RUNE(t \cdot t', g, s) = \RUNE(t, g', s')
\]
where:
\begin{eqnarray*}
%g' & = & \lambda c n. g(c, n \cdot t')\\
%s' & = & \lambda x. \ITE{x = \varepsilon}{\RUNE(t', g, s)}{s(x)}
g'(c,n) & = & g(c, n \cdot t')\\
s'(x) & = & \ITE{x = \varepsilon}{\RUNE(t', g, s)}{s(x)}
\end{eqnarray*}

\subsubsection{Priority choice.}

There is no continuation here: just a deterministic
choice between the two branches.
The second branch is taken if and only if the first branch deadlocks
in the current context.

\[
\RUNE(t \succ t', g, s) =
	\ITE{r \neq \diewith{\delta}}{r}{\RUNE(t', g, s)}
	\;\;\;\WHERE\;\;\;
   r = \RUNE(t, g, s))
\]

\subsubsection{Empty filter.}
This internal construct is introduced to ease the definition
of the loops. Intuitively, it forbids the core $t$ to terminate 
immediately.
\[
\RUNE(t\setminus\varepsilon, g, s) = \RUNE(t, g, s')
\]
where:
\begin{eqnarray*}
s'(x) & = & \ITE{x = \varepsilon}{\diewith{\deadlock}}{s(x)}
\end{eqnarray*}


\subsubsection{Priority loop.}

The semantics is defined according to the equivalence:
\begin{eqnarray*}
t^* & \Leftrightarrow & (t\setminus\varepsilon)\cdot t^* \succ \varepsilon
\end{eqnarray*}

\subsubsection{Catch.}
Note that $z$ is a catchable exception (either $\delta$ or a user-defined
exception).
\[
\RUNE(\catchdo{t}{z}{t'}, g, s) = \RUNE(t, g', s')
\]
where:
\begin{eqnarray*}
g'(c,n) & = & g(c, \catchdo{n}{z}{t'})\\
s'(x) & = & \ITE{x = z}{\RUNE(t', g, s)}{s(x)}
\end{eqnarray*}

\subsubsection{Parallel composition.}
\[
\RUNE(\merge{t}{t'}, g, s) = \RUNE(t, g', s')
\]
where:
\begin{eqnarray*}
s'(x) & = & \ITE{x = \varepsilon}{\RUNE(t', g, s)}{s(x)}\\
g'(c,n) & = & \RUNE(t', g'', s'') \;\;\;\mbox{with:}\\
s''(x) & = & \ITE{x = \varepsilon}{g(c,n)}{s(x)}\\
g''(c',n') & = & g(c \wedge c', \merge{n}{n'})
\end{eqnarray*}

\subsubsection{Weighted choice.}

\newcommand{\SORTE}{\mbox{$\mbox{\it Sort}_e$}}

The evaluation of the weights, and the (random) total ordering
of the branches according those actual weights are both 
performed by the environment:\\
$\SORTE(t_1/w_1, \cdots, t_n/w_n)$ returns:
\begin{itemize}
\item a priority expression $t_{\sigma(1)} \succ \cdots \succ t_{\sigma(k)}$ 
reflecting the priorities that have been (randomly) assigned 
to the branches; note that $k$ may be less than $n$, since some branches may have
an actual weight of $0$.  
\item the deadlock expression $\diewith{\delta}$ if all weights
are evaluated to $0$.
\end{itemize}
See~\S\ref{random-sort} for the precise definition of \SORTE.

\[
\RUNE(|_{i=1}^n\;\;t_i/w_i, g, s) = 
\RUNE(\SORTE(t_1/w_1, \cdots, t_n/w_n), g, s)
\]

\subsubsection{Random loop.}

The semantics is defined according to the equivalence:
\begin{eqnarray*}
\WLOOP{t}{i}
& \Leftrightarrow &
(t\setminus\varepsilon)\cdot \WLOOP{t}{i+1} / \omega_c(i)
\;\;\;|\;\;\;
\varepsilon / \omega_s(i)
\end{eqnarray*}

%%%%%%%%%%%%%%%%%%%%%%%%%%%%%%%%%%%%%%%%%%%%%%%%%%%%%%%%%%%%%%%%%%%%%%%%%
\subsection{The execution environment}

\subsubsection{Random sort of weighted choices}
\label{random-sort}

\subsection{Predefined loop profiles}
\label{loop-profiles}
