\subsection*{Entering scratch writing mode}

Press \key{s} to enter scratch writing; the cursor is modified and
you must click somewhere on the page to start writing text
there. Before clicking, you can
\begin{citemize}
 \item press \key{?} to get help,
 \item press \key{\char94 G} to quit scratching immediately,
 \item press \key{Esc}
 to enter the scratch writing settings mode and tune the font and font size.
\end{citemize}

\subsection*{Survival command kit when scratch writing}

{\ActiveDVI} recognizes the following keystrokes when scratch writing
on the page.

\noindent
\begin{tabularx}{\linewidth}{clcX}
\ikey{\char94 G}{quit}{End of scratch writing.}
\ikey{Esc}{settings}{Enter the scratch writing settings mode.}
\end{tabularx}

In the scratch writing settings mode, the cursor is modified and you
can set some charateristics of the scratch writing facility.
When in doubt, press

\begin{citemize}
 \item press \key{?} to get help,
 \item press \key{\char94 G} to quit scratching immediately,
 \item press \key{Esc} to quit the setting mode.
\end{citemize}
\Stretch

\newpage

\subsection*{Scratch writing settings mode keys}

When in the scratch writing settings mode, the following keys have the
following respective meanings:

\noindent
\begin{tabularx}{\linewidth}{clcX}
\ikey{$>$}{greater}{Increments the scratch font size.}
\ikey{$<$}{smaller}{Decrements the scratch font size.}
\ikey{b}{blue}{Set the color of the font to blue.}
\ikey{c}{cyan}{Set the color of the font to cyan.}
\ikey{g}{green}{Set the color of the font to green.}
\ikey{k}{black}{Set the color of the font to black.}
\ikey{m}{magenta}{Set the color of the font to magenta.}
\ikey{r}{red}{Set the color of the font to red.}
\ikey{w}{white}{Set the color of the font to white.}
\ikey{y}{yellow}{Set the color of the font to yellow.}
\ikey{B}{more blue}{Increment the blue component of the color.}
\ikey{G}{more green}{Increment the green component of the current color.}
\ikey{R}{more red}{Increment the red component of the current color.}
\ikey{$+$}{positive increment}{Set the color increment to positive.}
\ikey{$-$}{negative increment}{Set the color increment to negative.}
\ikey{$?$}{help}{Give the list of settings available.}
\ikey{Esc}{quit}{Quit the sratch writing settings mode.}
\end{tabularx}

\subsection*{Setting the scratching font size}

Just press \key{Esc} to enter the scratch writing settings mode, then
\key{$>$} or \key{$<$} to increment or decrement the font size; then
press \key{Esc} again, to leave the scratch writing settings mode and
continue to write on the page with the new font size.
\Stretch
